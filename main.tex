\documentclass[a4paper,12pt]{article}
\usepackage{jheppub} % for details on the use of the package, please see the JINST-author-manual
\usepackage{lineno}
\usepackage{indentfirst}
% \linenumbers



% \arxivnumber{1234.56789} % if you have one

\title{QIC Final Project: Anisotropic Transmission of quantum information through quantum fields}

% Collaborations

%% [A] If main author
%% \collaboration{\includegraphics[height=17mm]{collabroation-logo}\\[6pt]
%%  XXX collaboration}

%% or
%% [B] If "on behalf of"
%% \collaboration[c]{on behalf of XXX collaboration}


% Authors
% The "\note" macro will give a warning: "Ignoring empty anchor...", you can safely ignore it.

%% [A] simple case: 2 authors, same institution
%% \author[1]{A. Uthor\note{Corresponding author.}}
%% \author{and A. Nother Author}
%% \affiliation{Institution,\\Address, Country}

%% or, e.g.
%% [B] more complex case: 4 authors, 3 institutions, 2 footnotes
%% \author[a,b]{F. Irst,\note{Now at another university}}
%% \author[c]{S. Econd,}
%% \author[a,2]{T. Hird\note{Also at Some University.}}
%% \author[c,2]{and Fourth}
%% \affiliation[a]{Institution_1,\\Address, Country}
%% \affiliation[b]{Institution_2,\\Address, Country}
%% \affiliation[c]{Institution_3,\\Address, Country}

\author{T. Hsu}
\affiliation{National Taiwan University,\\
Taipei, Taiwan}
% \affiliation{Another University,\\
% different-address, Country}

% E-mail addresses: only for the corresponding author
\emailAdd{b11901097@ntu.edu.tw}

\abstract{In this letter, we briefly review the possible way to transmit the quantum information via quantum fields \cite{PhysRevD.101.036014}, and then we discuss }



\begin{document}
\maketitle
\flushbottom
% \section{Path Integral in Euclidean Signature}
\section{Quantum Channel: Via Quantum Mechanics}
In quantum information theory, the information is represented by a qubit, and it can be transformed, projected, and transmitted based on basic quantum mechanics posulates.
In this letter, we focus on the transmission of a qubit from a spacetime emitter Alice $ A $ to a receiver Bob $ B $.

There are various ways to transmit a qubit without contacting, which are based on the \textit{resources} Alice and Bob share.
For instance, if an entagled state is shared, they can transmit the qubit by Alice performing the Bell measurement and then send the result (a classical cbit) to Bob, which is the well-known \textit{quantum teleportation}.
Here, we simply consider transmisstion by a third quantum bit $ C $, $ \hat \rho_{ \tx{third}, 0 } $.
Denote Alice's qubit as $ \hat \rho_{ A, 0 } $ and Bob's qubit $ \hat \rho_{ B, 0 }$; the transmission is done by performing swap between $ A $ and $ C $, and then between $ C $ and $ B $. 
The whole process is unitary and does not violate the non-cloning process because Alice's qubit becomes $ \hat \rho_{ \tx{third}, 0 } $.


\section{Quantum Channel: Via Quantum Fields}
\subsection{Brief Review on Quantum Field Theory}
\subsection{Unruh-DeWitt model}
\appendix

\acknowledgments

% Bibliography

%% [A] Recommended: using JHEP.bst file
%% \bibliographystyle{JHEP}
%% \bibliography{biblio.bib}

%% or
%% [B] Manual formatting (see below)
%% (i) We suggest to always provide author, title and journal data or doi:
%% in short all the informations that clearly identify a document.
%% (ii) please avoid comments such as "For a review'', "For some examples",
%% "and references therein" or move them in the text. In general, please leave only references in the bibliography and move all
%% accessory text in footnotes.
%% (iii) Also, please have only one work for each \bibitem.
\bibliographystyle{plain}
\bibliography{ref}

% Example reference entry in ref.bib
% @article{example_reference,
%   author = {Author Name},
%   title = {Example Title},
%   journal = {Example Journal},
%   year = {2023},
%   volume = {1},
%   pages = {1--10}
% }
\end{document}
