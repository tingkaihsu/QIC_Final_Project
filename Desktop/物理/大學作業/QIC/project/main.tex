\documentclass[a4paper,11pt]{article}
\usepackage{jheppub} % for details on the use of the package, please see the JINST-author-manual
\usepackage{lineno}
\linenumbers



\arxivnumber{1234.56789} % if you have one

\title{\boldmath A title with some math: $x=1$}

% Collaborations

%% [A] If main author
%% \collaboration{\includegraphics[height=17mm]{collabroation-logo}\\[6pt]
%%  XXX collaboration}

%% or
%% [B] If "on behalf of"
%% \collaboration[c]{on behalf of XXX collaboration}


% Authors
% The "\note" macro will give a warning: "Ignoring empty anchor...", you can safely ignore it.

%% [A] simple case: 2 authors, same institution
%% \author[1]{A. Uthor\note{Corresponding author.}}
%% \author{and A. Nother Author}
%% \affiliation{Institution,\\Address, Country}

%% or, e.g.
%% [B] more complex case: 4 authors, 3 institutions, 2 footnotes
%% \author[a,b]{F. Irst,\note{Now at another university}}
%% \author[c]{S. Econd,}
%% \author[a,2]{T. Hird\note{Also at Some University.}}
%% \author[c,2]{and Fourth}
%% \affiliation[a]{Institution_1,\\Address, Country}
%% \affiliation[b]{Institution_2,\\Address, Country}
%% \affiliation[c]{Institution_3,\\Address, Country}

\author{A. Uthor}
\affiliation{One University,\\
some-street, Country}
\affiliation{Another University,\\
different-address, Country}

% E-mail addresses: only for the corresponding author
\emailAdd{first@one.univ}

\abstract{Abstract...}



\begin{document}
\maketitle
\flushbottom

\section{Some examples}
\label{sec:intro}

For internal references use label-refs: see section~\ref{sec:intro}.
Bibliographic citations can be done with "cite": refs.~\cite{a,b,c}.
When possible, align equations on the equal sign. The package
\texttt{amsmath} is already loaded. See \eqref{eq:x}.
\begin{equation}
\label{eq:x}
\begin{aligned}
x &= 1 \,,
\qquad
y = 2 \,,
\\
z &= 3 \,.
\end{aligned}
\end{equation}
Also, watch out for the punctuation at the end of the equations.


If you want some equations without the tag (number), please use the available
starred-environments. For example:
\begin{equation*}
x = 1
\end{equation*}

\section{Figures and tables}

All figures and tables should be referenced in the text and should be
placed on the page where they are first cited or in
subsequent pages. Positioning them in the source file
after the paragraph where you first reference them usually yield good
results. See figure~\ref{fig:i} and table~\ref{tab:i} for layout examples. 
Please note that a caption is mandatory  and it must be placed at the bottom of both figures and tables.

\begin{figure}[htbp]
\centering
\includegraphics[width=.4\textwidth]{example-image-a}
\qquad
\includegraphics[width=.4\textwidth]{example-image-b}
\caption{Always give a caption.\label{fig:i}}
\end{figure}

\begin{table}[htbp]
\centering
\begin{tabular}{lr|c}
\hline
x&y&x and y\\
\hline
a & b & a and b\\
1 & 2 & 1 and 2\\
$\alpha$ & $\beta$ & $\alpha$ and $\beta$\\
\hline
\end{tabular}
\caption{We prefer to have top and bottom borders around the tables.\label{tab:i}}
\end{table}

We discourage the use of inline figures (e.g. \texttt{wrapfigure}), as they may be
difficult to position if the page layout changes.

We suggest not to abbreviate: ``section'', ``appendix'', ``figure''
and ``table'', but ``eq.'' and ``ref.'' are welcome. Also, please do
not use \texttt{\textbackslash emph} or \texttt{\textbackslash it} for
latin abbreviaitons: i.e., et al., e.g., vs., etc.


\paragraph{Up to paragraphs.} We find that having more levels usually
reduces the clarity of the article. Also, we strongly discourage the
use of non-numbered sections (e.g.~\texttt{\textbackslash
  subsubsection*}).  Please also consider the use of
``\texttt{\textbackslash texorpdfstring\{\}\{\}}'' to avoid warnings
from the \texttt{hyperref} package when you have math in the section titles.



\appendix
\section{Some title}
Please always give a title also for appendices.





\acknowledgments

This is the most common positions for acknowledgments. A macro is
available to maintain the same layout and spelling of the heading.

\paragraph{Note added.} This is also a good position for notes added
after the paper has been written.


% Bibliography

%% [A] Recommended: using JHEP.bst file
%% \bibliographystyle{JHEP}
%% \bibliography{biblio.bib}

%% or
%% [B] Manual formatting (see below)
%% (i) We suggest to always provide author, title and journal data or doi:
%% in short all the informations that clearly identify a document.
%% (ii) please avoid comments such as "For a review'', "For some examples",
%% "and references therein" or move them in the text. In general, please leave only references in the bibliography and move all
%% accessory text in footnotes.
%% (iii) Also, please have only one work for each \bibitem.

\begin{thebibliography}{99}

\bibitem{a}
Author,
\emph{Title},
\emph{J. Abbrev.} {\bf vol} (year) pg.

\bibitem{b}
Author,
\emph{Title},
arxiv:1234.5678.

\bibitem{c}
Author,
\emph{Title},
Publisher (year).

\end{thebibliography}
\end{document}
